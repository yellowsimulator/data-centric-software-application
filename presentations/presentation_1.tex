%%%%%%%%%%%%%%%%%%%%%%%%%%%%%%%%%%%%%%%%%
% Beamer Presentation
% LaTeX Template
% Version 1.0 (10/11/12)
%
% This template has been downloaded from:
% http://www.LaTeXTemplates.com
%
% License:
% CC BY-NC-SA 3.0 (http://creativecommons.org/licenses/by-nc-sa/3.0/)
%
%%%%%%%%%%%%%%%%%%%%%%%%%%%%%%%%%%%%%%%%%

%----------------------------------------------------------------------------------------
%	PACKAGES AND THEMES
%----------------------------------------------------------------------------------------

\documentclass{beamer}

\mode<presentation> {

% The Beamer class comes with a number of default slide themes
% which change the colors and layouts of slides. Below this is a list
% of all the themes, uncomment each in turn to see what they look like.

%\usetheme{default}
%\usetheme{AnnArbor}
%\usetheme{Antibes}
%\usetheme{Bergen}
%\usetheme{Berkeley}
%\usetheme{Berlin}
%\usetheme{Boadilla}
%\usetheme{CambridgeUS}
%\usetheme{Copenhagen}
%\usetheme{Darmstadt}
%\usetheme{Dresden}
%\usetheme{Frankfurt}
%\usetheme{Goettingen}
%\usetheme{Hannover}
%\usetheme{Ilmenau}
%\usetheme{JuanLesPins}
%\usetheme{Luebeck}
\usetheme{Madrid}
%\usetheme{Malmoe}
%\usetheme{Marburg}
%\usetheme{Montpellier}
%\usetheme{PaloAlto}
%\usetheme{Pittsburgh}
%\usetheme{Rochester}
%\usetheme{Singapore}
%\usetheme{Szeged}
%\usetheme{Warsaw}

% As well as themes, the Beamer class has a number of color themes
% for any slide theme. Uncomment each of these in turn to see how it
% changes the colors of your current slide theme.

%\usecolortheme{albatross}
%\usecolortheme{beaver}
%\usecolortheme{beetle}
%\usecolortheme{crane}
%\usecolortheme{dolphin}
%\usecolortheme{dove}
%\usecolortheme{fly}
%\usecolortheme{lily}
%\usecolortheme{orchid}
%\usecolortheme{rose}
%\usecolortheme{seagull}
%\usecolortheme{seahorse}
%\usecolortheme{whale}
%\usecolortheme{wolverine}

%\setbeamertemplate{footline} % To remove the footer line in all slides uncomment this line
%\setbeamertemplate{footline}[page number] % To replace the footer line in all slides with a simple slide count uncomment this line

%\setbeamertemplate{navigation symbols}{} % To remove the navigation symbols from the bottom of all slides uncomment this line
}

\usepackage{graphicx} % Allows including images
\usepackage{booktabs} % Allows the use of \toprule, \midrule and \bottomrule in tables

%----------------------------------------------------------------------------------------
%	TITLE PAGE
%----------------------------------------------------------------------------------------

\title[Data management course, USN]{Developing Data Centric Software Application} % The short title appears at the bottom of every slide, the full title is only on the title page

\author{Yapi Donatien Achou} % Your name
\institute[Semcon] % Your institution as it will appear on the bottom of every slide, may be shorthand to save space
{
Semcon \\ % Your institution for the title page
\medskip
\textit{yapi-donatien.achou@semcon.com} % Your email address
}
\date{\today} % Date, can be changed to a custom date

\begin{document}

\begin{frame}
\titlepage % Print the title page as the first slide
\end{frame}

\begin{frame}
\frametitle{Overview} % Table of contents slide,
\tableofcontents % T
\end{frame}

%----------------------------------------------------------------------------------------
%	PRESENTATION SLIDES
%----------------------------------------------------------------------------------------
%------------------------------------------------
\section{Section 1: The ubiquity of data} 
%------------------------------------------------
%------------------------------------------------
\begin{frame}
\frametitle{The ubiquity of data}
\end{frame}
%------------------------------------------------




%------------------------------------------------
\section{Section 2: Data ingestion}
%------------------------------------------------
%------------------------------------------------
\begin{frame}
\frametitle{SECTION 2 FRAME 1 TITLE}
\end{frame}
%------------------------------------------------




%------------------------------------------------
\section{Section 3: Data ingestions}
%------------------------------------------------
%------------------------------------------------
\begin{frame}
\frametitle{SECTION 3 FRAME 1 TITLE}
\end{frame}
%------------------------------------------------




%------------------------------------------------
\section{Section 4: Data modelling}
%------------------------------------------------
%------------------------------------------------
\begin{frame}
\frametitle{SECTION 4 FRAME 1 TITLE}
\end{frame}
%------------------------------------------------




%------------------------------------------------
\section{Section 5: Data Modelling}
%------------------------------------------------
%------------------------------------------------
\begin{frame}
\frametitle{SECTION 5 FRAME 1 TITLE}
\end{frame}
%------------------------------------------------




%------------------------------------------------
\section{Section 6}
%------------------------------------------------
%------------------------------------------------
\begin{frame}
\frametitle{SECTION 6 FRAME 1 TITLE}
\end{frame}
%------------------------------------------------




%------------------------------------------------
% REFERENCES
%------------------------------------------------
\begin{frame}
\frametitle{References}
\footnotesize{
\begin{thebibliography}{99} % Beamer does not support BibTeX so references must be inserted manually as below
\bibitem[Smith, 2012]{p1} John Smith (2012)
\newblock Title of the publication
\newblock \emph{Journal Name} 12(3), 45 -- 678.
\end{thebibliography}
}
\end{frame}

%------------------------------------------------

\begin{frame}
\Huge{\centerline{The End}}
\end{frame}

%----------------------------------------------------------------------------------------

\end{document} 